\chapter{Results and Discussion}

\section{Magnetic Field Analysis}
In a Helmholtz coil configuration, where two identical coils are separated by a distance equal to their radius, the magnetic field between the coils is highly uniform around the midpoint, enhancing precision in experiments requiring consistent magnetic fields.

\section{Inductance Calculation}
We have calculated the inductance of the coil using the formula \( L = \frac{\mu_0 N^2 A}{l} \). The inductance of one coil is calculated \( L = 0.63 \)mH and the inductance of the two coils in series is calculated as \( L = 1.26 \)mH.

\section{Experimental Observations}

The measurement of the inductance of the coil was done using a simple RL circuit. The inductance of the coil was calculated to be 1.53mH. Calculation system is also verified by measuring a known inductor of 1mH.

\section{Waveform Analysis}

\subsection{Period is More Than Time Constant}
The period of the waveform is more than the time constant, which means that the inductor has enough time to charge and discharge fully. The voltage across the inductor shows a complete exponential rise and fall.

\subsection{Period is Equal to Time Constant}
The period of the waveform is equal to the time constant, which means that the inductor has just enough time to charge and discharge fully. The voltage across the inductor shows partial exponential rise and fall

\newpage
\thispagestyle{plain}

\subsection{Period is Less Than Time Constant}
The period of the waveform is less than the time constant, which means that the inductor does not have enough time to charge and discharge fully. The voltage across the inductor barely shows any change
